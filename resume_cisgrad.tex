\documentclass{resume} % Use the custom resume.cls style

\usepackage{CJKutf8}
\usepackage[left=0.4 in,top=0.4in,right=0.4 in,bottom=0.4in]{geometry} % Document margins
\newcommand{\tab}[1]{\hspace{.2667\textwidth}\rlap{#1}} 
\newcommand{\itab}[1]{\hspace{0em}\rlap{#1}}
\name{Xuefeng Jiang} % Your name
% You can merge both of these into a single line, if you do not have a website.
\address{+86 18908026296 \\ Beijing, China} 
\address{\href{mailto:jiangxf0903@gmail.com}{jiangxf0903@gmail.com} \\  \href{https://github.com/Sprinter1999}{https://github.com/Sprinter1999}}  %

\begin{document}
\begin{CJK}{UTF8}{gbsn}

%----------------------------------------------------------------------------------------
%----------------------------------------------------------------------------------------
%	EDUCATION SECTION
%----------------------------------------------------------------------------------------

\begin{rSection}{Education}
{\bf Institute of Computing Technology, Chinese Academy of Sciences}, Ph.D Student  \hfill {09/2021 - Present}\\
Major: Computer Science and Technology

Lab: Network Technology Research Center \& State-Key Lab, supervised by \href{http://www.ict.cas.cn/sourcedb_2018_ict_cas/cn/jssrck/200909/t20090917_2496676.html}{Prof.Liu} and \href{http://www.ict.cas.cn/sourcedb_2018_ict_cas/cn/jssrck/201612/t20161205_4716449.html}{Associate Prof.Wang}.

GPA: \textbf{3.82}/4.00

Key Course: Pattern Recognition \& Machine Learning (3.8), Computer Network (4.0), Multi-Media (4.0), etc

{\bf Beijing University of Posts and Telecommunications}, Bachelor's Degree  \hfill {09/2017 - 07/2021}\\
Major: Network Engineering, School of Computer Science

GPA: 3.69/4.00 (Rank top 3\%)

Award: Enterprise Scholarship (Top 10 out of 600), Outstanding Graduate of Beijing, etc

Key Course: Operating System(92), Algorithm(94), Web Development(90), Software Engineering(96), etc
%Minor in Linguistics \smallskip \\
%Member of Eta Kappa Nu \\
%Member of Upsilon Pi Epsilon \\



\end{rSection}

%----------------------------------------------------------------------------------------
% TECHINICAL STRENGTHS	
%----------------------------------------------------------------------------------------
% \begin{rSection}{SKILLS}
% \begin{tabular}{ @{} >{\bfseries}l @{\hspace{6ex}} l }
% Languages & C/C++, Python, Java, JavaScript   \\
% Frameworks & Pytorch, SpringBoot, Django \\
% Tools & Git, Docker\\
% \\
% Soft Skills & Time Management, Teamwork, Communication, Problem Solving, Leadership, Accountability
% \\
% \end{tabular}\\
% \end{rSection}



%----------------------------------------------------------------------------------------
% Projects
%----------------------------------------------------------------------------------------
\begin{rSection}{Research Experience}
\vspace{-1.25em}
\item \textbf{Academic Publication} \hfill {8/2022}
\begin{itemize}
    \itemsep -3pt {} 
     \item Focus on the joint area of distributed MLsys (or Federated Learning, FL) and weakly-supervised learning.
     \item It's my first work which got accepted at \textbf{ACM/CIKM 22'} full paper track.
     \item Towards more robust performance against noisy-labeled data with relatively less computation cost.
    \item The code is open-sourced at Github and receives 13 stars up now.
 \end{itemize}
\item \textbf{R\&D Project with Huawei MindSpore Team towards FL}  \hfill {10/2020 - 10/2021}
\begin{itemize}
    \itemsep -3pt {} 
     \item Reviewed a certain amount of papers related to FL.
     \item Reproduced some baselines for client-edge-cloud FL system and Asynchronous FL Optimization.
    \item Contributed to develop a holistic FL framework for research incorporating various optimization mechanisms, security techniques and a customizable data partition module. Some codes are adopted by MindSpore project.
    \item Other information remains secret for related cooperation policy : ) .
 \end{itemize}
\end{rSection} 

%----------------------------------------------------------------------------------------
\begin{rSection}{Other Activities} 
\begin{itemize}
    \item 	\textbf{Opensource Contribution}: Previously worked as a opensource community volunteer at \href{https://fedml.ai/}{FedML}, a start-up at UCLA. Currently contribute to a awesome paperlist opensource project \href{https://github.com/youngfish42/Awesome-Federated-Learning-on-Graph-and-Tabular-Data}{Awesome-FL} with 312 stars up now.
    \item	\textbf{Development projects}: Image Manipulation Forensics (Pytorch), Hotel backend management system(Java), CPU function implement on a experimental chip(VHDL), C-Assembly  Compiler(C/C++), Travelling Scheduling Router(Python/JavaScript), High-concurrency online shopping system(Golang), a interacting shell for human and embedded system(C), etc.
\end{itemize}


\end{rSection}

%------------------------
% Use this more detailed section if you have Relevant work experience
% keep your resume to 1 page, if you need to remove a project to display relevant experience
% that is okay
% ----------------------------
% \begin{rSection}{EXPERIENCE}

% \textbf{Role Name} \hfill Jan 2017 - Jan 2019\\
% Company Name \hfill \textit{San Francisco, CA}
%  \begin{itemize}
%     \itemsep -3pt {} 
%      \item Achieved X\% growth for XYZ using A, B, and C skills.
%      \item Led XYZ which led to X\% of improvement in ABC
%     \item Developed XYZ that did A, B, and C using X, Y, and Z. 
%  \end{itemize}
 
% \textbf{Role Name} \hfill Jan 2017 - Jan 2019\\
% Company Name \hfill \textit{San Francisco, CA}
%  \begin{itemize}
%     \itemsep -3pt {} 
%      \item Achieved X\% growth for XYZ using A, B, and C skills.
%      \item Led XYZ which led to X\% of improvement in ABC
%     \item Developed XYZ that did A, B, and C using X, Y, and Z. 
%  \end{itemize}

% \end{rSection} 

\begin{rSection}{Work Experience}
\vspace{-1.25em}
\item \textbf{Algorithm Engineer Intern} {@Ai Startup Team from Tsinghua University} \hfill 08/2020 - 11/2020

\begin{itemize}
    \item 	Contributed to knowledge extraction from a large corpus of mathematics tests, and knowledge graph based knowledge management of mathematics concepts to serve for downstream tasks(eg. Auto inference and solution to mathematics problems).
\end{itemize}

\item \textbf{Quality Assurance Intern} {@YuanFuDao Inc.} \hfill 08/2019 - 10/2019
\begin{itemize}
    \item 	Tested company's inner online management web frontend \& backend functions.
\end{itemize}
\end{rSection} 

%----------------------------------------------------------------------------------------


\end{CJK}
\end{document}
